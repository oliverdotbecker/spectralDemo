%% Erläuterungen zu den Befehlen erfolgen unter
%% diesem Beispiel.

\documentclass[11pt]{scrartcl}

\usepackage[utf8]{inputenc}
\usepackage[T1]{fontenc}
\usepackage{lmodern}
\usepackage[ngerman]{babel}
\usepackage{amsmath}
\usepackage{graphicx}
\usepackage{subcaption}
\usepackage{wrapfig}
\usepackage{float}
\usepackage[printonlyused, footnote]{acronym}
\usepackage[
    colorlinks=true,
    urlcolor=blue,
    linkcolor=black,
    citecolor=black
]{hyperref}
\usepackage[font={scriptsize},figurename=Bild]{caption}

\title{MASTER THESIS}
\author{Oliver Becker}
\date{31. Oktober 2019} %%\today
\begin{document}

\begin{center}
    \center\includegraphics[width=0.4\linewidth]{images/logo.png}
    \paragraph{}
    \huge MASTER THESIS
\end{center}
\vspace{3cm}\noindent
{\large Title:  Abgleich von LED Lichtquellen auf Basis einer komplexen Lichtsteuerung}
\par\vspace{3cm}\noindent
Submitted by:               Oliver Becker\\
\\
Matriculation number:       725810\\
\par\vspace{2cm}\noindent
1st Academic Supervisor:    Prof. Dr. Herbert Krauß\\
\\
2nd Academic Supervisor:    Prof. Dr. Shun-Ping Chen\\
\\
Industrial Supervisor:      Dipl.-Ing. Hartmut Cordes\\
\par\vspace{2cm}\noindent
Completion Date:            31.10.2019
\clearpage

\tableofcontents
\clearpage

%%%%%%%%%%%%%%%%%%%%%%%%%% Vorwort %%%%%%%%%%%%%%%%%%%%%%%%%%%%%%%%%%%%%%%%%%%%%%%%%%%%%%%%%%%%%%%%
\section*{Vorwort}
\addcontentsline{toc}{section}{Vorwort}
Weltweit finden große Veranstaltungen statt. Bei allen Inszenierungen spielt das Licht eine 
große Rolle. Egal, ob einfach nur hell oder szenisch ausgeleuchtet, bedarf es einer Steuerung 
für diese Lichttechnik. Bühnen und Ansprüche der Veranstalter werden immer größer und komplexer. 
So geht es über das einfache An- und Ausschalten eines Scheinwerfers schon sehr weit hinaus. 
Heutzutage werden Lichtanlagen mit leistungsfähigen Computersystemen gesteuert, welche von 
geschultem Personal bedient werden müssen.\\
In der Filmindustrie spielt Licht eine große Rolle. Die Ausleuchtung eines Film-Sets entscheidet 
auch wie die Handlung vom Betrachter wahrgenommen wird. Dies gilt sowohl für statische Szenen, 
als auch für die zumeist hoch Aufwändigen dynamischen Installationen.
Eine der zurzeit größten Herausforderungen stellt die korrekte Farbwiedergabe der mannigfaltigen 
Lichtquellen an einem Film-Set dar.
Dies wird vor allem durch die unterschiedenen chromatischen Eigenschaften von den Licht 
emittierenden Quellen erschwert. Diese Quellen sind heute zu xx\% LEDs.\\
Als Beispiel wird hier ein Film-Set für eine Motorrad Verfolgungsszene genommen. Hier wird im 
Hintergrund mit großen LED Kacheln der computergenerierte Hintergrund abgespielt. Mit kleinen 
LED Streifen links und rechts neben dem im Studio aufgebauten Motorrad wird dynamisch Licht auf 
den Schauspieler emittiert, um eine größere Dynamik zu simulieren.
Zusätzlich wird mit kleinen Bildschirmen weiterer computergenerierter Hintergrund von vorne auf 
das Motorrad und den Schauspieler projiziert, um realistische Spiegelungen in der 
Windschutzscheibe und der Brille des Schauspielers zu erhalten.\\
Alle diese aufgezählten Emitter haben zumeist alle unterschiedlichste chromatische Kennwerte 
sofern diese überhaupt bekannt sind. Ein Abgleich dieser Quellen ist wünschenswert für ein 
homogeneres Farbbild.

%%%%%%%%%%%%%%%%%%%%%%%%%% Erklärung %%%%%%%%%%%%%%%%%%%%%%%%%%%%%%%%%%%%%%%%%%%%%%%%%%%%%%%%%%%%%%%
\section*{Erklärung}
\addcontentsline{toc}{section}{Erklärung}
Ich versichere hiermit, dass ich die vorliegende Arbeit selbständig verfasst und keine anderen als die
im Literaturverzeichnis angegebenen Quellen benutzt habe.\\
Alle Stellen, die wörtlich oder sinngemäß aus veröffentlichten oder noch nicht veröffentlichten
Quellen entnommen sind, sind als solche kenntlich gemacht.\\
Die Zeichnungen oder Abbildungen in dieser Arbeit sind von mir selbst erstellt worden oder mit
einem entsprechenden Quellennachweis versehen.\\
Diese Arbeit ist in gleicher oder ähnlicher Form noch bei keiner anderen Prüfungsbehörde
eingereicht worden.\\
\par\vspace{2cm}\noindent
Darmstadt, den 30.10.2019 \hspace*{0pt}\hfill Oliver Becker
\clearpage

%%%%%%%%%%%%%%%%%%%%%%%%%% Abstrakt %%%%%%%%%%%%%%%%%%%%%%%%%%%%%%%%%%%%%%%%%%%%%%%%%%%%%%%%%%%%%%%%
\section*{Abstrakt}
\addcontentsline{toc}{section}{Abstrakt}

\section*{Abstract}
\addcontentsline{toc}{section}{Abstract}
\clearpage

%%%%%%%%%%%%%%%%%%%%%%%%%% MA Lighting Technology %%%%%%%%%%%%%%%%%%%%%%%%%%%%%%%%%%%%%%%%%%%%%%%%%%
\section{MA Lighting Technology}
MA Lighting Technology GmbH (im Folgenden MA) wurde 1983 von Michael Adenau zusammen mit
den drei Mitgesellschaftern Ernst Ebrecht, Thomas Stanger und Werner Hauptvogel gegründet. MA
entwickelt und produziert Lichtsteuerungen für die Bereiche Fernsehen, Veranstaltungen, Theater
und szenische Gebäudebeleuchtung. Schon Anfang der 1990er Jahre erkannten die Firmengründer
das Potenzial der elektrischen und digitalen Steuerung von Scheinwerfern und bauten die ersten
Lichtstellpulte, sowie die zugehörigen Komponenten. Im Laufe der Jahre entwickelte sich MA zu
einem der international führenden Unternehmen für computergesteuerte Lichtstellpulte,
Netzwerkkomponenten und digitale Dimmer Systeme.\\
In der Referenzliste von MA findet man viele große internationale Theater und Opernhäuser, Bands
von AC/DC über Sting bis U2, die Eröffnungsfeier der Olympischen Spiele, der Eurovision Song
Contest 2019 oder die Lichtinstallation zum 25. Jahrestag des Falls der Berliner Mauer 2014.\\
\begin{wrapfigure}{r}{0.5\textwidth}
    \vspace{-25pt}
    \begin{center}
        \includegraphics[width=0.48\textwidth]{images/csm_MA_building_WBB.jpg}
    \end{center}
    \vspace{-20pt}
    \caption{Firmensitz Waldbüttelbrunn}
    \vspace{-15pt}
\end{wrapfigure}
Der Hauptsitz von MA Lighting Technology GmbH befindet sich in Waldbüttelbrunn in der Nähe von
Würzburg. Dort beschäftigt MA circa 100 Mitarbeiter. In den fünf Gebäuden sitzen Soft- und
Hardwareentwicklung, Testabteilung, Produktion, Lager und Management. Alle Produkte werden
größtenteils von Hand von den Mitarbeitern vor Ort zusammengebaut. Platinen und Gehäuse werden
von externen Firmen geliefert. Die einzelnen Bauteile werden von den Mitarbeitern auf den
Platinen aufgesteckt und verlötet. Anschließend werden die bestückten Platinen in den Gehäusen
montiert und verkabelt. Nach einem mehrtägigen BurnIn-Test und einer Werkskonfiguration werden
die Geräte verpackt und sind versandfertig.\\
\\
Kundensupport und Vertrieb werden in Deutschland von der externen Firma MA Lighting International
durchgeführt. Diese hat ihren Sitz in Paderborn. Die Konsultation der Mitarbeiter in
Waldbüttelbrunn erfolgt lediglich bei größeren technischen Fragen und Problemen.\\
Eines der ersten Produkte von MA war der Lightcommander24, eine analoge Lichtsteuerkonsole auf
Basis des \ac{dmx} Protokolls von 1990. Nach kurzer Zeit folgten Dimmer für
Scheinwerfer und \ac{dmx}-Demultiplexer, um analoge Geräte ansteuern zu können.\\

\begin{figure}[H]
    \centering
    \begin{subfigure}[b]{0.45\textwidth}
        \includegraphics[width=\textwidth]{images/grandMA_1.png}
        \caption{grandMA 1 Full Size}
    \end{subfigure}
    \hfill 
    \begin{subfigure}[b]{0.25\textwidth}
        \includegraphics[width=\textwidth]{images/MA_2Port_Node_onPC_13019x.png}
        \caption{2port Node}
    \end{subfigure}
    \caption{grandMA 1 Serie Produkte}\label{fig:gma1}
\end{figure}
\noindent
Ebenso entwickelte MA die Steuerpulte Scan-Commander und die grandMA Series 1 sowohl als
Konsole, als auch als Computerapplikation. Das Besondere an dieser Technologie ist, dass \ac{dmx}-Daten
nicht nur über \ac{xlr}, sondern auch über ein eigenes Ethernet-Netzwerkprotokoll namens 'MA-Net'
ausgegeben werden. Als Gegenstück am anderen Ende des Netzwerkkabels existieren sogenannte
\ac{dmx}-Nodes. Diese fungieren als Protokollwandler zwischen MA-Net und \ac{dmx}.\\
Die Firma setzte immer wieder Standards in der Branche. Im Zuge des technischen Fortschritts –
Einsatz von LEDas als Lichtquelle, moderne Netzwerktechnologie und Videodatenverarbeitung –
wurde die Produktpalette wesentlich erweitert. Hinzu kam die grandMA2 Series.\\
\begin{figure}[H]
    \centering
    \begin{subfigure}[b]{0.45\textwidth}
        \includegraphics[width=\textwidth]{images/MA_grandMA2_full-size.png}
        \caption{grandMA 2 Full Size}
    \end{subfigure}
    \hfill 
    \begin{subfigure}[b]{0.4\textwidth}
        \includegraphics[width=\textwidth]{images/MA-VPU-plus_d.png}
        \caption{VPU}
    \end{subfigure}
    \caption{grandMA 2 Serie Produkte}\label{fig:gma2}
\end{figure}
\noindent
Die grandMA2 Series ist die moderne Fortführung der grandMA Series 1. Außerdem stellt sie
folgende Produkte bereit: Die Network Processing Unit (NPU), eine Prozessoreinheit, welche die
Konsole in ihren Berechnungen unterstützt, indem die Konsole Rechenaufträge auf die NPU dezentral
auslagert. Die Video Processing Unit (VPU) verarbeitet Video-Content und gibt diesen nativ über
Graphikkarten oder Pixelmapping mit Hilfe von \ac{dmx} aus. Die Replay-Unit (RPU) ist eine vollwertige
Konsole ohne die typischen Steuerelemente, sie dient beispielsweise als Havariegerät. Zuletzt noch
verschiedene Variationen der Nodes, welche weiterhin als Protokollwandler dienen.\\
Alle diese Produkte adressieren vor allem den Highend-Profibereich für Theaterproduktionen,
Liveshows und Festinstallationen.\\
\begin{figure}[H]
    \includegraphics[width=\textwidth]{images/grandMA3-full-size.png}
    \caption{grandMA3 Full Size}\label{fig:gma3}
\end{figure}
\noindent
Seit Mitte 2017 entwickelt MA eine neue Produkt Serie. Die grandMA3 Serie. Diese Serie greift die
in den letzten 34 Jahren gesammelte Erfahrung auf, jedoch wurde jede Zeile Code neu geschrieben,
um die technischen Möglichkeiten der neuen Hardware vollumfänglich nutzen zu können.\\
Neben vielen Veränderungen gegenüber der Vorserien arbeitet die Farblogik der grandMA3 Software
mit den Spezifikationen des \ac{cie} 1931 Standards.

\clearpage

%%%%%%%%%%%%%%%%%%%%%%%%%% Grundlagen %%%%%%%%%%%%%%%%%%%%%%%%%%%%%%%%%%%%%%%%%%%%%%%%%%%%%%%%%%%%%%%%%%%%
\section{Grundlagen}

In dieser Arbeit werden Lichtquellen mit der grandMA3 Applikation von MA Lighting über ein Protokoll
mit der Bezeichnung \ac{dmx} angesteuert.\\
Für eine erste Überprüfung geeigneter Hardware wurde mit einer kleinen Demo Applikation gearbeitet. Diese wurde
in JavaScript und \ac{html} mit Hilfe des Electron Frameworks erstellt.\\
Weiter wird diese Applikation für die Ausgabe von \ac{dmx} genutzt.\\
Im folgenden werden die hier verwendeten Protokolle, Programme, und Frameworks dargelegt.

\subsection{DMX}
Diese Abkürzung steht für Digital Multiplex. Dahinter verbirgt sich ein digitales Steuerprotokoll,
welches vornehmlich in der Veranstaltungstechnik eingesetzt wird. Es wurde zuerst als DMX-
512/1990 standardisiert. Im November 2004 wurde dieser Standard aktualisiert, bekannt als
DMX512-A (ANSI E1.11-2008). \ac{dmx} basiert elektrisch auf dem seriellen RS-485 (EIA-485)
Schnittstellenstandard.\\
\begin{figure}[H]
    \includegraphics[width=\textwidth]{images/dmx.png}
    \caption{Physikalische Übertragung DMX-512/1990 (RS-485)}
\end{figure}
\noindent
Das eigentliche Signal wird dabei seriell auf einem Leitungspaar symmetrisch übertragen auf der
einen Leitung mit invertierten und auf der anderen mit nichtinvertiertem Pegel. Das Signal ist so
weniger störungsempfindlich, da sich externe Einstreuungen auf beide Datenleitungen gleichmäßig
auswirken und am Empfänger nicht das Pegelniveau, sondern die Pegeldifferenz ausgewertet wird.
Das ermöglicht laut Standard [2] bis zu 1200m lange Übertragungsstrecken und eine
Datenübertragungsrate bis zu 12 \ac{MBps}. Ein Netzwerk aus mehreren Geräten wird als Bus aufgebaut.\\
Die Datenübertragung erfolgt mittels einer Universal Asynchronous Receiver Transmitter (UART)
gesteuerten asynchronen seriellen Schnittstelle. Ein Datenframe besteht aus acht Datenbits und zwei
Stoppbits. Die Symbolrate beträgt 250 kBaud (Baud = Bits pro sec).
\begin{figure}[H]
    \includegraphics[width=\textwidth]{images/DMX_zeitlich.png}
    \caption{Zeitliche Darstellung \ac{dmx}-Signal}
\end{figure}
\noindent
\begin{table}[H]
    \small
    Legende zu Bild 6:\\
    \begin{tabular*}{\textwidth}{l @{\extracolsep{\fill}} llllll}
        Nr. & Signalabschnitt & Min & Soll & Max & Einheit\\
        1 & BREAK & 88 & 176 & - & $\mu$s\\
        2 & MARK after BREAK & 8 $\mu$s & - & < 1 s\\
        3 & Slot Time & 43,12 & 44,0 & 44,88 & $\mu$s\\
        4 & Startbit & 3,92 & 4,0 & 4,08 & $\mu$s\\
        5 & LSB (niederwertigstes Datenbit) & 3,92 & 4,0 & 4,08 & $\mu$s\\
        6 & MSB (höchstwertiges Datenbit) & 3,92 & 4,0 & 4,08 & $\mu$s\\
        7 & Stoppbit & 3,92 & 4,0 & 4,08 & $\mu$s\\
        8 & MARK Inter-Slot Time & 0 & - & < 1 & s\\
        9 & MARK before BREAK & 0 & - & < 1 & s\\
        10 & BREAK – BREAK (Übertragungsausfall) & 1,2 ms & - & 1,25 s & s\\
    \end{tabular*}
\end{table}
\noindent
Ein \ac{dmx}-Paket beginnt mit mindestens 88 $\mu$s (22 Bitlängen) niedrigem Pegel (logisch 0) – dieser
Abschnitt wird 'BREAK' genannt. Durch ihn wird eine einfache Erkennung des Paketanfangs
ermöglicht, da quasi jeder handelsübliche UART den Break als ungültiges Datenbyte mit fehlenden
Stoppbits meldet. Darauf folgt 'MARK after BREAK' mit mindestens 8 $\mu$s (2 Bitlängen) hohem Pegel /
Ruhezustand des Daten-Busses (logisch 1). In dieser 'MARK'-Zeit können sich langsamer getaktete
Controller auf ein neues \ac{dmx}-Paket einstellen. Dann wird das Startbyte mit dem Wert 0 für \ac{dmx}
übertragen. Das Startbyte ist der Slot 0. Anschließend werden die anderen 512 Slots mit den
relevanten \ac{dmx}-Daten gesendet. Es können, müssen aber nicht alle 513 Slots übertragen werden.
Eine Adressierung der Slots ist jedoch nicht möglich – der zweite gesendete Slot ist für den ersten
Kanal, der dritte Slot für den zweiten Kanal etc.. Sollte die Übertragung zu einem beliebigen
Zeitpunkt unterbrochen werden, kann sie durch das Senden eines neuen \ac{dmx}-Paketes wieder
aufgenommen werden. Die 'BREAK'-Sequenz führt automatisch zu einem Zurücksetzen aller noch
offenen Übertragungen.\\
\noindent
In der Spezifikation DMX512-A wurden erstmals Maximalzeiten für die Slot Time (3) und den
zeitlichen Abstand zwischen zwei BREAKs (10) definiert. Wenn die Zeit zwischen zwei BREAKs größer
als eine Sekunde ist, wertet ein Empfänger dies als Übertragungsausfall. Ein Übertragungsausfall ist
folgendermaßen definiert: 'Ein Übertragungsausfall kann dann angenommen werden, wenn der
angeschlossene Empfänger nicht innerhalb der angegebenen Zeit einen gültigen BREAK mit
darauffolgendem NULL-Startcode erkennen kann.' Weiter wurde eine empfohlene Länge des
BREAKs definiert. Dies soll den Empfängern die Dekodierung erleichtern.
\clearpage

\subsection{Art-Net}
Art-Net ist ein Netzwerkprotokoll, dass zur Verteilung von \ac{dmx} und \ac{rdm} Daten genutzt werden kann.
Es ist in OSI Layer 4 auf UDP aufgebaut. Das Protokoll ist von der Firma Artistic Licence Holdings Ltd. lizenziert.\\
Hier wird Art-Net in der dritten Version verwendet. Diese Version wurde 2011 publiziert.\\
Über das Art-Net 3 Protokoll können bis zu 32.768 \ac{dmx} Universen verwaltet werden.\\
\cite{artnet}\\
Das Protokoll kann sowohl Unicast, Multicast oder Brodcast eingesetzt werden. Somit können auch komplexe Netzwerke
in einer physikalischen Struktur aufgebaut werden.\\
\\
Mit Hilfe von Protokollwandlern wie dem MA xPort Node kann Art-Net in \ac{dmx} umgewandelt werden. Auch von anderen
Herstellern gibt es auf dem Markt weit verbreitete Protokollwandler.\\
\\
Art-Net IP Adressen sind nach folgendem Muster aufgebaut:
\begin{table}[H]
    \begin{tabular*}{\textwidth}{l @{\extracolsep{\fill}} lllll}
        Bytes & 4 & 3 & 2 & 1\\
        IP Adresse & Netz & Netz & Host & Host\\
        Subnetz & 255 & 255 & 0 & 0\\
    \end{tabular*}
\end{table}
\noindent
In der Regel werden IP Adressen für Art-net mit dem Netz 2.X.X.X oder 10.X.X.X verwendet. Art-Net wird über den
Netzwerkport 6454 gesendet und empfangen.

Ein Art-Net Paket kann verschiedene Informationen haben, neben \ac{dmx} Daten. Hier sendet die Applikation jedoch
lediglich ArtDmx Pakete. Diese Pakete transportieren DMX Daten. Hier ist der Aufbau des ArtDMX Paketes dargelegt.\\
\begin{table}[H]
    \begin{tabular*}{\textwidth}{l @{\extracolsep{\fill}} lll}
        Byte Offset & Name & Größe & Beschreibung\\
        0 & ID & 8 & Art-Net\\
        8 & OpCodeLo & 1 & Pakettyp (ArtDmx) LSB\\
        9 & OpCodeHi & 1 & Pakettyp (ArtDmx) MSB\\
        10 & ProtVer & 2 & Protokollversion (14)\\
        12 & Sequenz & 1 & Paketsequenznummer\\
        13 & Physical & 1 & Physikalischer Port\\
        14 & Universe & 2 & Art-Net Universum\\
        16 & Length & 2 & Länge der DMX Daten\\
        18 & Data[] & 2-512 & DMX Daten\\
    \end{tabular*}
\end{table}
\cite{artdmx}
\clearpage

\subsection{HTML,JavaScript,CSS}
\ac{html} ist eine Struktur-Beschreibungs Sprache. Sie ist an den \ac{xml} Standard angelehnt. \ac{html}
wird hauptsächlich in der Webentwicklung verwendet. Doch kann \ac{html} auch für Webgestützte Desktop
Applikationen genutzt werden.\\
Die Sprache enthält Grundlegende Elemente zur Beschreibung von Tabellen, einbinden von Medieninhalten sowie
das einbinden von weiteren HTML Dateien.\\
\\
\ac{css} ist eine in der \ac{json} Notation geschrieben. Hier kann auf Elemente in einem \ac{html} Baum
verwiesen werden, und mit verschiedenen Attributen das Erscheinungsbild im Webbrowser, oder in der Applikation,
geändert werden.\\
Einfache Dinge wie Hintergrund- oder Textfarben, bis hin zu komplexen Animationen kann mit \ac{css} beschrieben
werden.\\
\\
JavaScript ist eine Skript Sprache die wie \ac{css} meist in der Webentwicklung verwendet wird. Jedoch kann
mit einer entsprechenden JavaScript Laufzeitumgebung auch ohne Browser auf Endgeräten ausgeführt werden.
\clearpage

\subsection{Electron}
Electron ist ein Framework welches zum erstellen von Desktop Anwendungen genutzt werden kann. Electron funktioniert
wie ein Webbrowser. Die Applikation zeigt \ac{html} Inhalte und andere Textinhalte an.\cite{electron} Der Programmierer kann diese
Browserumgebung nach belieben modifizieren und erweitern. Electron wurde aus dem Projekt Atom Editor von Github entwickelt.\\
Das Framework basiert auf dem Chromium Projekt von Google. Auf dem Chromium Projekt basiert auch der Chrome Webbrowser von
Google.\\
Electron bietet eine Betriebssystem übergreifende Applikation als Resultat. Electron Applikationen können unter Windows, Linux und
MacOS ausgeführt werden.\\
Electron ist Open Source und wird von einer breiten Community weltweit permanent weiterentwickelt.\cite{electron}\\
Das Projekt wird in der Programmiersprachen C++ und JavaScript geschrieben.\\
\\
Der JavaScript anteil wird mit NodeJS auf den Zielcomputern ausgeführt. NodeJS ist eine JavaScript Laufzeitumgebung. Laufzeitumgebungen
sind zum ausführen eines Programms in einer Sprache zwingend erforderlich.

\subsection{Git und Github}
Git ist eine Datei Versionierungs- und Management Software. Git wird häufig in der Software Entwicklung mit mehreren Stationen eingesetzt.\\
Eine Instanz von einem mit Git verwalteten Dateisystem wird Repository bezeichnet. Diese Instanz wird meist auf einem Server bereitgestellt.
Ein Nutzer kann den Ihnalt des Repositories auf sein Endgerät abrufen.
Abgerufene Dateien, vornehmlich nicht binäre Textdateien, können anschließend lokal auf einem Endgerät editiert werden. Wenn eine neue Version
der Datei in das Repository eingestellt werden soll, werden die Änderungen an den Dateien mit einem Kommentar eingestellt (committed).
Diese Eingestellten Änderungen werden dann mit dem Server synchronisiert und mit den Änderungen von anderen Stationen automatisch zusammengeführt.\\
\\
Dieses System macht das Entwickeln einer Software mit mehreren Entwicklern deutlich einfacher.\\
\\
Github ist eine Webplattform. Dort wurde auch das Git Repository der Testapplikation dieses Projektes gehostet.
\clearpage

%%%%%%%%%%%%%%%%%%%%%%%%%% Lichtquellen %%%%%%%%%%%%%%%%%%%%%%%%%%%%%%%%%%%%%%%%%%%%%%%%%%%%%%%%%%%%%%%
\section{Lichtquellen}
TODO Add something about light sources
\clearpage

%%%%%%%%%%%%%%%%%%%%%%%%%% Farbräume %%%%%%%%%%%%%%%%%%%%%%%%%%%%%%%%%%%%%%%%%%%%%%%%%%%%%%%%%%%%%%%%%%
\section{Farbräume}
TODO Add something about color spaces
\clearpage

%%%%%%%%%%%%%%%%%%%%%%%%%% Findung eines Ansatzes %%%%%%%%%%%%%%%%%%%%%%%%%%%%%%%%%%%%%%%%%%%%%%%%%%%%%
\section{Findung eines Ansatzes}
Licht ist elektromagnetischen Spektrum messbar. Somit ist ein Farbton eine Frequenz und eine Helligkeit eine Amplitude. Dieser Teil des
elektromagnetischen Spektrums lässt dich zum Beispiel mit Photodioden messen.\\
\\
Es gibt einige Hersteller von entsprechenden Spektralsensoren für sichtbares Licht. Einer davon ist AMS. In dieser Arbeit wurden mehrere
Sensoren dieser Firma untersucht.\\
\clearpage

%%%%%%%%%%%%%%%%%%%%%%%%%% Testapplikation und Sensorik %%%%%%%%%%%%%%%%%%%%%%%%%%%%%%%%%%%%%%%%%%%%%%
\section{Testapplikation und Sensorik}
\clearpage

%%%%%%%%%%%%%%%%%%%%%%%%%% Messungen %%%%%%%%%%%%%%%%%%%%%%%%%%%%%%%%%%%%%%%%%%%%%%%%%%%%%%%%%%%%%%%%%
\section{Messungen}
\clearpage

%%%%%%%%%%%%%%%%%%%%%%%%%% Automatisierung %%%%%%%%%%%%%%%%%%%%%%%%%%%%%%%%%%%%%%%%%%%%%%%%%%%%%%%%%%%
\section{Automatisierung}
\clearpage

%%%%%%%%%%%%%%%%%%%%%%%%%% Anhang %%%%%%%%%%%%%%%%%%%%%%%%%%%%%%%%%%%%%%%%%%%%%%%%%%%%%%%%%%%%%%%%%%%%
\section{Anhang}
\subsection{Abbildungsverzeichnis}
\listoffigures
\clearpage

\subsection{Abkürzungsverzeichnis}
\begin{acronym}
    \acro{cie}[CIE]{Commission internationale de l’éclairage}
    \acro{css}[CSS]{Cascading Style Sheets}
    \acro{dmx}[DMX]{Digital Multiplex}
    \acro{html}[HTML]{Hypertext Markup Language}
    \acro{json}[JSON]{JavaScript Object Notation}
    \acro{MBps}[MBps]{Megabits pro Sekunde}
    \acro{rdm}[RDM]{Remote Device Management}
    \acro{xlr}[XLR]{drei oder fünf polige Steckverbindung}
    \acro{xml}[XML]{Extensible Markup Language}
\end{acronym}
\clearpage

\subsection{Literaturverzeichnis}
\begin{thebibliography}{xxxxxxxxxxxxxxxxxxx}
    \bibitem[Art-Net Homepage, 2019]{artnet}"'Home - Art-Net"', https://art-net.org.uk/, 07.08.2019
    \bibitem[Art-Net Documatation, 2019]{artdmx}"'ArtDmx – Packet Definition - Art-Net"', https://art-net.org.uk/structure/streaming-packets/artdmx-packet-definition/, 10.08.2019
    \bibitem[Electron Homepage, 2019]{electron}"'Electron | Plattformübergreifende Desktop-Anwendungen mit JavaScript, HTML und CSS entwickeln."', https://electronjs.org/, 08.08.2019
 \end{thebibliography}
\clearpage

\section{\LaTeX Beispiele}
\begin{align}
E &= mc^2                 \\
m &= \frac{m_0}{\sqrt{1-\frac{v^2}{c^2}}}
\end{align}
\\
Sonderzeichen
\url{https://de.wikibooks.org/wiki/LaTeX-Kompendium:_Sonderzeichen}\\
micro - $\mu$\\
\\
Lines
\url{https://www.namsu.de/latex/kapitel6.html}\\
\\
Embed figures
\url{https://en.wikibooks.org/wiki/LaTeX/Floats,_Figures_and_Captions}\\
\\
Sub and Top Text
\textsuperscript{top} and \textsubscript{sub}\\
\\
Fußnoten
\footnote{Text der Fußnote}
\clearpage

\end{document}