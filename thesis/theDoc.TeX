%% Erläuterungen zu den Befehlen erfolgen unter
%% diesem Beispiel.

\documentclass[11pt]{scrartcl}

\usepackage[utf8]{inputenc}
\usepackage[T1]{fontenc}
\usepackage{lmodern}
\usepackage[ngerman]{babel}
\usepackage{amsmath}
\usepackage{graphicx}
\usepackage{wrapfig}
\usepackage[
    colorlinks=true,
    urlcolor=blue,
    linkcolor=black
]{hyperref}

\title{MASTER THESIS}
\author{Oliver Becker}
\date{31. Oktober 2019} %%\today
\begin{document}

\begin{center}
    \center\includegraphics[width=0.4\linewidth]{logo.png}
    \paragraph{}
    \huge MASTER THESIS
\end{center}
\vspace{2cm}\noindent
{\large Title:  Abgleich von LED Lichtquellen auf Basis einer komplexen Lichtsteuerung}
\par\vspace{4cm}\noindent
Submitted by:   Oliver Becker\\
\par\vspace{2cm}\noindent
1st Academic Supervisor:    Dr. Prof. Herbert Krauß\\
\\
2nd Academic Supervisor:    Dr. Prof. Shun-Ping Chen\\
\\
Industrial Supervisor:      Dipl.-Ing. (TU) Hartmut Cordes\\
\par\vspace{2cm}\noindent
Completion Date:    31.10.2019
\clearpage

\tableofcontents
\clearpage

\section{Vorwort}
Weltweit finden große Veranstaltungen statt. Bei allen Inszenierungen spielt das Licht eine 
große Rolle. Egal, ob einfach nur hell oder szenisch ausgeleuchtet, bedarf es einer Steuerung 
für diese Lichttechnik. Bühnen und Ansprüche der Veranstalter werden immer größer und komplexer. 
So geht es über das einfache An- und Ausschalten eines Scheinwerfers schon sehr weit hinaus. 
Heutzutage werden Lichtanlagen mit leistungsfähigen Computersystemen gesteuert, welche von 
geschultem Personal bedient werden müssen.\\
In der Filmindustrie spielt Licht eine große Rolle. Die Ausleuchtung eines Film-Sets entscheidet 
auch wie die Handlung vom Betrachter wahrgenommen wird. Dies gilt sowohl für statische Szenen, 
als auch für die zumeist hoch Aufwändigen dynamischen Installationen.
Eine der zurzeit größten Herausforderungen stellt die korrekte Farbwiedergabe der mannigfaltigen 
Lichtquellen an einem Film-Set dar.
Dies wird vor allem durch die unterschiedenen chromatischen Eigenschaften von den Licht 
emittierenden Quellen erschwert. Diese Quellen sind heute zu xx\% LEDs.\\
Als Beispiel wird hier ein Film-Set für eine Motorrad Verfolgungsszene genommen. Hier wird im 
Hintergrund mit großen LED Kacheln der Computergenerierte Hintergrund abgespielt. Mit kleinen 
LED Streifen links und rechts neben dem im Studio aufgebauten Motorrad wird dynamisch Licht auf 
den Schauspieler emittiert, um eine größere Dynamik zu simulieren.
Zusätzlich wird mit kleinen Bildschirmen weiterer Computergenerierter Hintergrund von vorne auf 
das Motorrad und den Schauspieler projiziert, um realistische Spiegelungen in der 
Windschutzscheibe und der Brille des Schauspielers zu erhalten.\\
Alle diese aufgezählten Emitter haben zumeist alle unterschiedlichste chromatische Kennwerte 
sofern diese überhaupt bekannt sind. Ein Abgleich dieser Quellen ist wünschenswert für ein 
homogeneres Farbbild. 
\clearpage

\section{Abstrakt}
\section{Abstract}
\clearpage

\section{Einführung}
MA Lighting Technology GmbH (im Folgenden MA) wurde 1983 von Michael Adenau zusammen mit
den drei Mitgesellschaftern Ernst Ebrecht, Thomas Stanger und Werner Hauptvogel gegründet. MA
entwickelt und produziert Lichtsteuerungen für die Bereiche Fernsehen, Veranstaltungen, Theater
und szenische Gebäudebeleuchtung. Schon Anfang der 1990er Jahre erkannten die Firmengründer
das Potenzial der elektrischen und digitalen Steuerung von Scheinwerfern und bauten die ersten
Lichtstellpulte, sowie die zugehörigen Komponenten. Im Laufe der Jahre entwickelte sich MA zu
einem der international führenden Unternehmen für computergesteuerte Lichtstellpulte,
Netzwerkkomponenten und digitale Dimmer Systeme.\\
In der Referenzliste von MA findet man viele große internationale Theater und Opernhäuser, Bands
von AC/DC über Sting bis U2, die Eröffnungsfeier der Olympischen Spiele, der Eurovision Song
Contest 2013 oder die Lichtinstallation zum 25. Jahrestag des Falls der Berliner Mauer 2014.\\
Der Hauptsitz von MA Lighting Technology GmbH
befindet sich in Waldbüttelbrunn in der Nähe von
Würzburg. Dort beschäftigt MA circa 75 Mitarbeiter. In
den drei Gebäuden sitzen Soft- und Hardwareentwicklung,
Testabteilung, Produktion, Lager und
Management. Alle Produkte werden größtenteils von
Hand von den Mitarbeitern vor Ort zusammengebaut.
Platinen und Gehäuse werden von externen Firmen
geliefert. Die einzelnen Bauteile werden von den
Mitarbeitern auf den Platinen aufgesteckt und verlötet.
Anschließend werden die bestückten Platinen in den Gehäusen montiert und verkabelt. Nach einem
mehrtägigen BurnIn-Test und einer Werkskonfiguration werden die Geräte verpackt und sind
versandfertig.\\
Kundensupport und Vertrieb werden von der externen Firma "Lightpower" durchgeführt. Diese hat
ihren Sitz in Paderborn. Die Konsultation der Mitarbeiter in Waldbüttelbrunn erfolgt lediglich bei
größeren technischen Fragen und Problemen.\\
Eines der ersten Produkte von MA war der Lightcommander24, eine analoge Lichtsteuerkonsole auf
Basis des DMX (Digital Multiplex) Protokolls von 1990. Nach kurzer Zeit folgten Dimmer für
Scheinwerfer und DMX-Demultiplexer, um analoge Geräte ansteuern zu können.\\
\\
Bild 2 - grandMA Series 1 Full Size 2port Node
Bild 1 - Firmensitz Waldbüttelbrunn [14]\\
\\
Ebenso entwickelte MA die Steuerpulte Scan-Commander und die grandMA Series 1 sowohl als
Konsole, als auch als Computerapplikation. Das Besondere an dieser Technologie ist, dass DMX-Daten
nicht nur über XLR1, sondern auch über ein eigenes Ethernet-Netzwerkprotokoll namens "MA-Net"
ausgegeben werden. Als Gegenstück am anderen Ende des Netzwerkkabels existieren sogenannte
DMX-Nodes. Diese fungieren als Protokollwandler zwischen MA-Net und DMX.\\
Die Firma setzte immer wieder Standards in der Branche. Im Zuge des technischen Fortschritts –
Einsatz von LEDas als Lichtquelle, moderne Netzwerktechnologie und Videodatenverarbeitung –
wurde die Produktpalette wesentlich erweitert. Hinzu kam die grandMA2 Series.\\
Die grandMA2 Series ist die moderne Fortführung der grandMA Series 1. Außerdem stellt sie
folgende Produkte bereit: Die Network Processing Unit (NPU), eine Prozessoreinheit, welche die
Konsole in ihren Berechnungen unterstützt, indem die Konsole Aufträge auf die NPU dezentral
auslagert. Die Video Processing Unit (VPU) verarbeitet Video-Content und gibt diesen nativ über
Graphikkarten oder Pixelmapping mit Hilfe von DMX aus. Die Replay-Unit (RPU) ist eine vollwertige
Konsole ohne die typischen Steuerelemente, sie dient beispielsweise als Havariegerät. Zuletzt noch
verschiedene Variationen der Nodes, welche weiterhin als Protokollwandler dienen [1].\\
Alle diese Produkte adressieren vor allem den Highend-Profibereich für Theaterproduktionen,
Liveshows und Festinstallationen. Um auch Kunden im mittleren Marktsegment ansprechen zu
können, ist im April 2015 auf der international wichtigsten Messe der Branche, der Prolight\&Sound,
eine weitere Konsolenfamilie vorgestellt worden, die dot2. Zum weiteren Ausbau des
Systemgedankens beschäftigt sich MA momentan verstärkt mit Netzwerktechnologien und steht kurz
vor der Veröffentlichung eines eigenen MA-Ethernet-Switches.\\
In der neunwöchigen Berufspraktischen Phase 2 habe ich mit den Arbeiten am DMX-Recorder
begonnen. Dieser basiert auf der Plattform der aktuellen Version des "MA 8Port Nodes onPC". Die
Recorderfunktion wurde als Erweiterung entworfen. In dieser Phase wurden die Grundfunktionen
des Recorders implementiert und die Hardware entwickelt.\\
Die Arbeiten im Zeitraum der Bachelorarbeit greifen die vorherigen Ergebnisse auf, daher sind diese
hier ebenfalls beschrieben. Die Funktionen des DMX-Recorders wurden um die Features Spulen,
Playlists, Schleifen und eine Agenda erweitert. Des Weiteren wurde ein umfangreiches Webinterface
mit verschiedenen Zugriffsrechten entwickelt, mit welchem sowohl das eigentliche Basisgerät als
auch die Recorderfunktionen ferngesteuert werden können.\\
Zunächst werde ich auf den Begriff DMX eingehen. Anschließend sind die Fähigkeiten des
Basisgerätes, sowie die Erweiterungen und Entwicklungen für den DMX-Recorder unterteilt in
Hardware und Software erläutert. Die Kapitel 2 bis 2.3 und 3 bis 3.3.3 waren im Bericht der
Berufspraktischen Phase 2 bereits enthalten und wurden teilweise ergänzt, um die Vollständigkeit
dieser Arbeit zu gewährleisten.\\

\subsection{DMX}
Diese Abkürzung steht für Digital Multiplex [2]. Dahinter verbirgt sich ein digitales Steuerprotokoll,
welches vornehmlich in der Veranstaltungstechnik eingesetzt wird. Es wurde zuerst als DMX-
512/1990 standardisiert. Im November 2004 wurde dieser Standard aktualisiert, bekannt als
DMX512-A (ANSI E1.11-2008). DMX basiert elektrisch auf dem RS-485 (EIA-485)
Schnittstellenstandard.\\
Das eigentliche Signal wird dabei seriell auf einem Leitungspaar symmetrisch übertragen auf der
einen Leitung mit invertierten und auf der anderen mit nichtinvertiertem Pegel. Das Signal ist so
weniger störungsempfindlich, da sich externe Einstreuungen auf beide Datenleitungen gleichmäßig
auswirken und am Empfänger nicht das Pegelniveau, sondern die Pegeldifferenz ausgewertet wird.
Das ermöglicht laut Standard [2] bis zu 1200m lange Übertragungsstrecken und eine
Datenübertragungsrate bis zu 12 Mbps. Ein Netzwerk aus mehreren Geräten wird als Bus aufgebaut
(siehe Bild 7).\\
Die Datenübertragung erfolgt mittels einer Universal Asynchronous Receiver Transmitter (UART)
gesteuerten asynchronen seriellen Schnittstelle. Ein Datenframe besteht aus acht Datenbits und zwei
Stoppbits. Die Symbolrate beträgt 250 kBaud (Baud = Bits pro sec).
Bild 5 - Physikalische Übertragung DMX-512/1990 (RS-485)\\
\\
Legende zu Bild 6 (S.17 [2]):\\
Nr. Signalabschnitt Min Soll Max Einheit\\
1 "BREAK" 88 176 - $\mu$s\\
2 "MARK after "BREAK 8 $\mu$s - < 1 s\\
3 Slote Time 43,12 44,0 44,88 $\mu$s\\
4 Startbitf 3,92 4,0 4,08 $\mu$s\\
5 LSB (niederwertigstes Datenbit) 3,92 4,0 4,08 $\mu$s\\
6 MSB (höchstwertigstes Datenbit) 3,92 4,0 4,08 $\mu$s\\
7 Stoppbitg 3,92 4,0 4,08 $\mu$s\\
8 "MARK" Inter-Slot Time 0 - < 1 s\\
9 "MARK" before "BREAK"h 0 - < 1 s\\
10 "BREAK" – "BREAK" (Übertragungsausfall) 1,2 ms - 1,25 s\\
\\
Ein DMX-Paket beginnt mit mindestens 88 $\mu$s (22 Bitlängen) niedrigem Pegel (logisch 0) – dieser
Abschnitt wird "BREAK" genannt. Durch ihn wird eine einfache Erkennung des Paketanfangs
ermöglicht, da quasi jeder handelsübliche UART den Break als ungültiges Datenbyte mit fehlenden
Stoppbits meldet. Darauf folgt "MARK after BREAK" mit mindestens 8 $\mu$s (2 Bitlängen) hohem Pegel /
Ruhezustand des Daten-Busses (logisch 1). In dieser "MARK"-Zeit können sich langsamer getaktete
Controller auf ein neues DMX-Paket einstellen. Dann wird das Startbytei mit dem Wert 0 für DMX
übertragen. Das Startbyte ist der Slot 0. Anschließend werden die anderen 512 Slots mit den
relevanten DMX-Daten gesendet. Es können, müssen aber nicht alle 513 Slots übertragen werden.
Eine Adressierung der Slots ist jedoch nicht möglich – der zweite gesendete Slot ist für den ersten
Kanal, der dritte Slot für den zweiten Kanal etc.. Sollte die Übertragung zu einem beliebigen
Zeitpunkt unterbrochen werden, kann sie durch das Senden eines neuen DMX-Paketes wieder
aufgenommen werden. Die "BREAK"-Sequenz führt automatisch zu einem Zurücksetzen aller noch
offenen Übertragungen.\\
\\
Bild 6 - Zeitliche Darstellung DMX-Signal
Bild 7 - Netzwerktopologien [13]\\
\\
In der Spezifikation DMX512-A wurden erstmals Maximalzeiten für die Slot Time (3) und den
zeitlichen Abstand zwischen zwei BREAKs (10) definiert. Wenn die Zeit zwischen zwei BREAKs größer
als eine Sekunde ist, wertet ein Empfänger dies als Übertragungsausfall. Ein Übertragungsausfall ist
folgendermaßen definiert: "Ein Übertragungsausfall kann dann angenommen werden, wenn der
angeschlossene Empfänger nicht innerhalb der angegebenen Zeit einen gültigen BREAK mit
darauffolgendem NULL-Startcode erkennen kann." [2] Weiter wurde eine empfohlene Länge des
BREAKs definiert. Dies soll den Empfängern die Dekodierung erleichtern.
\clearpage

\section{Lichtquellen}
\clearpage

\section{Farbräume}
TODO Add something about color spaces
\clearpage

\section{Sensorik}
\clearpage

\section{Messungen}
\clearpage

\section{Automatisierung}
\clearpage


\section{\LaTeX Beispiele}
\begin{align}
E &= mc^2                 \\
m &= \frac{m_0}{\sqrt{1-\frac{v^2}{c^2}}}
\end{align}
\\
Sonderzeichen
\url{https://de.wikibooks.org/wiki/LaTeX-Kompendium:_Sonderzeichen}\\
micro - $\mu$\\
\\
Lines
\url{https://www.namsu.de/latex/kapitel6.html}\\
\\
Embed figures
\url{https://en.wikibooks.org/wiki/LaTeX/Floats,_Figures_and_Captions}
\clearpage

\end{document}